\documentclass[twoside,12pt]{article}

\usepackage{hulipsum}
\renewcommand{\contentsname}{Tartalomjegyzék}
\renewcommand{\thefootnote}{\fnsymbol{footnote}}
\usepackage[magyar]{babel}
\usepackage{t1enc}
\usepackage[inner=3cm,outer=5cm,top=3cm,bottom=3cm,
bindingoffset=1cm,marginparwidth=3cm,marginparsep=0.5cm
,headheight=15pt]{geometry}
\usepackage{fancyhdr}
\usepackage{xcolor}

\usepackage[colorlinks,urlcolor=red,linktoc=all]{hyperref}



\begin{document}

\fancypagestyle{plain}{
\fancyfoot[C]{}
\fancyfoot[R]{\thepage}
\fancyhead[L,C,R]{}
}
\pagestyle{fancy}
\fancyhead[C]{}
\fancyhead[R]{\thepage}
\fancyfoot[C]{Miskolci egyetem}
\fancyhead[LE]{\leftmark}
\fancyhead[LO]{\rightmark}
\renewcommand{\footrulewidth}{0.4pt}

\title{\textbf{Tex 2.Gyakorlat}}
\author{\textit{1.feladat}}
\date{\today}
\maketitle

\autoref{s:szamozatlan}
\begin{abstract}
\hulipsum[1-2]

\footnote{labjegyzet}
\end{abstract}

\setcounter{tocdepth}{5}
\pagenumbering{roman}
\tableofcontents


\clearpage
\pagenumbering{arabic}
\section{elsoszakasz}
\autoref{a:quotation}\\
\autopageref{s:zagyva2}\\
\ref{a:quotation} 
\ref{p:paragraph} 
\ref{s:zagyva2}
\\
\pageref{a:quotation}
\pageref{p:paragraph}
\pageref{s:zagyva2}
\\
\href{https://www.uni-miskolc.hu/~viktoria.vadon/}{https://www.uni-miskolc.hu/~viktoria.vadon/}

\subsubsection{elsoelsoelsoszakasz}
\subsection{zagyva1}
\hulipsum[1-2]
\marginpar{Páratalan oldal}
\clearpage
\subsection{zagyva2}
\label{s:zagyva2}
\hulipsum[3-4]
\marginpar{Páros oldal}
\clearpage
\section[section2]{masodikszakasz}
\setcounter{secnumdepth}{5}
\subsection{subsection}
\subsubsection{subsubsection}
\paragraph{paragraph}
\label{p:paragraph}
\subparagraph{subparagraph}
\clearpage
\markboth{számozatlan}{}
\section*{számozatlan}
\label{s:szamozatlan}
\hulipsum:
\clearpage
\appendix
\section{harmadik}
\subsection{quote}
\begin{quote}
\hulipsum[15-16]
\end{quote}

\subsection{quotation}
\begin{quotation}
\label{a:quotation}
\hulipsum[20-21]
\end{quotation}
\clearpage
\section{negyedik}
\begin{verse}
\hulipsum[100-101]
\end{verse}
\subsection{kék}





\end{document}
