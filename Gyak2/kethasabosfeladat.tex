\documentclass[twoside,twocolumn]{article}
\setlength{\columnsep}{1cm}
\usepackage{hulipsum}
\renewcommand{\contentsname}{Tartalomjegyzék}
\renewcommand{\thefootnote}{\fnsymbol{footnote}}
\usepackage[magyar]{babel}
\usepackage{t1enc}
\usepackage[inner=3cm,outer=5cm,top=3cm,bottom=3cm,
bindingoffset=1cm,marginparwidth=3cm,marginparsep=0.5cm]{geometry}



\begin{document}
\pagestyle{headings}
\title{\textbf{Tex 2.Gyakorlat}}
\author{\textit{1.feladat}}
\date{\today}
\maketitle

\begin{abstract}
\hulipsum[1-2]

\footnote{labjegyzet}
\end{abstract}
\setcounter{tocdepth}{5}
\pagenumbering{roman}
\tableofcontents

\clearpage
\pagestyle{myheadings}
\markboth{Zsigó Bence}{kethasabosfeladat}
\pagenumbering{arabic}
\section{elsoszakasz}

\subsubsection{elsoelsoelsoszakasz}
\subsection{zagyva1}
\hulipsum[1-2]
\marginpar{Páratalan oldal}
\clearpage
\subsection{zagyva2}
\hulipsum[3-4]
\marginpar{Páros oldal}
\clearpage

\section[section2]{masodikszakasz}
\setcounter{secnumdepth}{5}
\subsection{subsection}
\subsubsection{subsubsection}
\paragraph{paragraph}
\subparagraph{subparagraph}
\appendix
\section{harmadik}
\subsection{quote}
\begin{quote}
\hulipsum[15-16]
\end{quote}

\subsection{quotation}
\begin{quotation}
\hulipsum[20-21]
\end{quotation}

\section{negyedik}
\begin{verse}
\hulipsum[100-101]
\end{verse}
\subsection{kék}





\end{document}