\documentclass{article}
\usepackage{amsmath}
\usepackage[magyar]{babel}
\usepackage{t1enc}
\usepackage{mathtools}
\usepackage{amsfonts}
\usepackage{enumitem}
\usepackage{xcolor}


\begin{document}
\section{Bevezető}
\begin{enumerate}[label=\alph*)]
\item Az \(\frac{1}{n^2}\) sor összege:
\[\sum_{n=1}^\infty \frac{1}{n^2} = \frac{\pi^2}{6}\]
\item Az \textit{n}! (\textit{n} faktoriális) a számok szorzata 1-től \textit{n}-ig, azaz
\begin{equation}
n! := \prod_{k=1}^n k=1\cdot 2\cdot ...\cdot n\end{equation}
Konvenció szerint 0! = 1.
\item  Legyen \(0\leq k\leq n\). A binomiális együttható
\[\binom{n}{k}:=\frac{n!}{k!\cdot (n-k!)}\text{,}\]
ahol a faktoriálist (\textcolor{red}{1}) szerint definiáljuk.
\item Az előjel- azaz szignum függvényt a következőképpen definiáljuk:
\[sgn(x):=\begin{cases}
1, & \text{ha } x > 0, \\
0, & \text{ha } x = 0, \\
-1, & \text{ha } x < 0, \\
\end {cases}\]
\end{enumerate}
\pagebreak
\section{Determináns}
\begin{enumerate}[label=\alph*)]
\item Legyen 
\[[n]:=\{1,2,\cdot\cdot\cdot ,n\}\]
a természetes számok halmaza 1-től \textit{n}-ig.
\item Egy \textit{n}-ed rendű \textit{permutáció $\sigma $} egy bijekció [\textit{n}]-ből [\textit{n}]-be. Az \textit{n}-ed rendű permutációk halmazát az ún. szimmetrikus csoportot, 
\(S_n\)-el jelöljük.
\item Egy $\sigma \in S_n$ permutációban inverziónak nevezünk egy \textit{(i, j)} párt, ha \textit{i < j}
de $\sigma_i > \sigma_j$.
\item Egy $\sigma \in S_n$ permutáció paritásának az inverziók számát nevezzük:
\[I(\sigma):=\Bigl\vert\{(i,j)\left.\right\vert i,j \in [n],i<j, \sigma_i < \sigma_j \}\Bigr\vert\]
\item Legyen $A \in \mathbb{R}^{nxn}$, egy \textit{n x n}-es (négyzetes) valós mátrix:
\[A= \left( \begin{matrix}
a_{11} & a_{12}& \cdots & a_{1n}\\
a_{21} & a_{22}& \cdots & a_{2n}\\
\vdots & \vdots & \ddots & \vdots \\
a_{n1} & a_{n2}& \cdots & a_{nn}\\
\end{matrix} \right)\]
Az A mátrix determinánsát a következőképpen definiáljuk:
\begin{equation}det(A)= \left\vert \begin{matrix}
a_{11} & a_{12}& \cdots & a_{1n}\\
a_{21} & a_{22}& \cdots & a_{2n}\\
\vdots & \vdots & \ddots & \vdots \\
a_{n1} & a_{n2}& \cdots & a_{nn}\\
\end{matrix} \right\vert 
:= \sum_{\sigma \in S_n} (-1)^{I(\sigma)}\prod_{i=1}^n a_{i\sigma_i}\end{equation}
\end{enumerate}

\pagebreak
\section{Logikai azonosság}
Tekintsük az L = {0, 1} halmazt, és rajta a következő, igazságtáblával definiált műveleteket:\\
\begin{center}

\begin{tabular}{lr||c|c|c}
x & y	& x $\vee$ y & x $\wedge$ y & x $\rightarrow$ y \\ \hline
0 & 0 & 0 & 0 & 1 \\
0 & 1 & 1 & 0 & 1 \\
1 & 0 & 1 & 0 & 0 \\
1 & 1 & 1 & 1 & 1 \\
\end{tabular}
\end{center}
Legyenek $a,b,c,d \in L$. Belátjuk a következő azonosságot:
\begin{equation}
a\wedge b\wedge c)\rightarrow d=a\rightarrow (b\rightarrow (c\rightarrow d))\end{equation}.
A következő azonosságokat bizonyítás nélkül használjuk:
\begin{subequations}
\begin{equation}
x\rightarrow x =\bar{x} \vee y
\end{equation}
\begin{equation}
\overline{x \vee y} = \bar{x} \wedge \bar{y}\text{\;\;\;\;\;\;\;} \overline{x \wedge y} = \bar{x} \vee \bar{y}
\end{equation}
\end{subequations}
A (\textcolor{red}{3}) bal oldala, (\textcolor{red}{4}) felhasználásával
\begin{equation}
(a\wedge b \wedge c)\rightarrow d \underset{(\textcolor{red}{4a})}{=} \overline{a\wedge b \wedge c}\vee b \underset{(\textcolor{red}{4b})}{=}(\bar{a}\vee \bar{b} \vee \bar{c})\vee d
\end{equation}
A (\textcolor{red}{3}) bal oldala, (\textcolor{red}{4a}) ismételt felhasználásával
\begin{align}
\nonumber a\rightarrow(b\rightarrow(c\rightarrow d))&=(\bar{a})\vee(b\rightarrow(c\rightarrow d))\\ 
&=(\bar{a})\vee(\bar{b} \vee (c\rightarrow d))\\ \nonumber
&=(\bar{a})\vee(\bar{b} \vee(\bar{c} \vee d))
\end{align}
6)
ami a $\vee$ asszociativitása miatt egyenlő (\textcolor{red}{5}) egyenlet
\pagebreak
\section{Binomiális tétel}
\begin{subequations}
\begin{align}
(a+b)^{n+1}
&= (a+b) \cdot \left( \sum_{k=0}^n \binom{n}{k} a^{n-k}b^k \right)\\
\nonumber &=\hdots\\
\begin{split}
&= \sum_{k=0}^n \binom{n}{k} a^{(n+1)-k}b^k
+ \sum_{k=1}^{n+1} \binom{n}{k-1} a^{(n+1)-k}b^{k}
\end{split}\\
\nonumber &=\hdots\\
\begin{split}
&= \binom{n+1}{0} a^{n+1-0} b^0
+ \sum_{k=1}^n \binom{n+1}{k} a^{(n+1)-k}b^k\\ 
&+ \binom{n+1}{n+1} a^{n+1-(n+1)} b^{n+1}
\end{split}\\
\begin{split}
&= \sum_{k=0}^{n+1} \binom{n+1}{k} a^{(n+1)-k}b^k
\end{split}
\end{align} 

\end{subequations}
\end{document}