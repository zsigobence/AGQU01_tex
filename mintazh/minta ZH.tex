\documentclass[twocolumn]{report}

%%%magyar nyelvi beállításoks
\usepackage[magyar]{babel}
\usepackage{t1enc}
\frenchspacing

%%%zagyva szöveg
\usepackage{hulipsum}

%%%fejléc, lábléc
\usepackage{fancyhdr}
%fancy oldalstílus beállítása
\fancyhead[R]{\thepage}
\fancyhead[L]{\nouppercase{\leftmark}}
\fancyfoot[C]{Minta ZH}
%plain oldalstílus újadefiniálása
\fancypagestyle{plain}{%
\fancyhf{}% %minden mező törlése
\fancyhead[R]{\thepage}}

%%%táblázathoz
\usepackage{array} %kód beszúrása az oszlophoz
\usepackage[table]{xcolor} %színezés
\usepackage{multirow} %cellák összevonása

%%matematika
\usepackage{amsmath} %%mindig jó ötlet
\usepackage{mathtools} %%pmatrix-hoz
\usepackage{amsfonts} %%\mathbb-hez
\usepackage{amsthm} %%tételekhez
\theoremstyle{definition} %%stílus: vastag betűs cím,  \normalfont törzs. az ez után kiadott \newtheorem parancsokra érvényes, amíg nem írjuk felül
\newtheorem{tet}{Tétel}
\newtheorem{defin}{Definíció} %%def-nek rövidebb lenne, de nem szerencsés, mert \def belső kulcsszó
%% nem varázsoltunk számlálókkal, mert a példában is egymástól és chapter-től is függetlenül számozódtak

%%pszeudokód
\usepackage{algpseudocode} %%kód írása
\usepackage{algorithm} %%úsztatás
\floatname{algorithm}{Algoritmus} %%átnevezés, magyarítás

%%programozás
%% \centerpair parancs, 2 kötelező argumentummal
\newcommand{\centerpair}[2]{%
\par\noindent% %%egész sor kell nekünk
\begin{minipage}[t]{0.48\linewidth}% %%bal oldali doboz, fél* oldalszélesség, függőlegesen felülre igazítva
%%kevesebb, mint fél, hogy középen ne folyjon össze a szöveg!
\begin{flushright}% %%jobbra zárás
#1% %%argumentum beillesztése
\end{flushright}%
\end{minipage}%
\hfill\vline\hfill% %% középvonal és az extra hely kitöltése körülötte középen -- enélkül is teljesítettük a feladatot
\begin{minipage}[t]{0.48\linewidth}%
\begin{flushleft}% %%balra zárás -- fölösleges, mert alapértelmezett
#2%
\end{flushleft}%
\end{minipage}%
\par}

\begin{document}
%%%oldalstílus
\pagestyle{fancy}

%%%címsor
\title{Zárthelyi dolgozat\\\Large Minta ZH}
\author{Vadon Viktória XXXXXX}
\date{\today}
\maketitle

\chapter{Oldalbeállítás}

\section{Első szakasz}

\hulipsum[1-12]

\section{Második szakasz}

\hulipsum[13-24]

\chapter{Táblázat}

\hulipsum[1-3]

%% >{oszlop elé beszúrandó kód}r
%% \columncolor{green%30}: green 30%, white 70%
%% N/A cellákhoz külön fehér háttérszín \cellcolor-ral
%% tartsuk fejben, hogy a multicolumn és multirow szintaxisa különbözik!
%% \multicolumn{oszlopszám}{igazítás}{tartalom}
%% \multirow{sorok száma}{szélesség}{tartalom}
%%	 ``sorok száma'' nem feltétlen egész szám, csak egy dobozmagasság, normál sormagasság hányszorosa
%%	 *-gal automatikus szélesség, de nem tördel

\begin{table*} %% bármely float *-os verziója mindkét hasábot kitölti kéthasábos dokumentumban, és oldal tetejére kerül
\caption{Egy vállalati kimutatás}
\begin{center} %% csak hogy szebb legyen
\begin{tabular}{c|c|>{\columncolor{green!30}}r>{\columncolor{red!30}}r>{\columncolor{yellow!30}}r}
év & ágazat & bevétel & kiadás & nettó bevétel \\ \hline
\multirow{4}{*}{2020} & marketing & xx & xx & xx \\
 & humán erőforrás & \multicolumn{3}{c}{N/A} \\
 & logisztika & xx & xx & xx \\
 & gyártás & xx & xx & xx \\ \hline
\multirow{4}{*}{2021} & marketing & \cellcolor{white}N/A & xx & \cellcolor{white}N/A \\
 & humán erőforrás & xx & xx & xx \\
 & logisztika & xx & xx & xx \\
 & gyártás & xx & \multicolumn{2}{c}{N/A}
\end{tabular}
\end{center}
\end{table*}

\hulipsum

\chapter{Matematika}

\begin{defin}[Mátrix szorzás] %% [] arguentum: definíció neve, ami a ()-ben jelenik meg
%% érdemes a szöveg másolni a pdf-ből
%% sőt, formulákhoz érdemes másolás-beillesztéssel újrahasznosítani kódrészleteket -- vagy akár saját parancsokat bevezetni, mint \newcommand{\redn}{\textcolor{red}{n}}
Legyenek $m,\textcolor{red}{n},r \in \mathbb{Z}^+$, $A \in\mathbb{R}^{m\times \textcolor{red}{n}}$, és $B\in\mathbb{R}^{\textcolor{red}{n}\times r}$. Bontsuk fel $A$-t sorvektoraival, B-t pedig oszlopevektoraival:
\begin{equation} %%egyenlet: számozott, egysoros matematikai mód
A = \begin{pmatrix} %%mátrix kerek zárójellel
a_1^T \\
%% a mátrix is array-re épül -- \cellcolor és társai használhatók benne
\cellcolor{red!30}\hphantom{a_2^T} a_2^T \hphantom{a_2^T}\\ 
%%\hphantom: láthatatlan szöveg -- vizuális trükk, ezzel szélesítjük ki a sort és a színezést
\vdots \\
a_m^T 
%% utolsó sor után nem kell \\
\end{pmatrix}, 
\quad %% térköz
B = \begin{pmatrix}
%% egyetlen sor kell, de üres cellákból megcsinálunk plusz fél-fél sort előtte és utána a beszínezéshez
%%	 A-ben is csinálhattuk volna ugyanezt, plusz üres oszlopok -- azaz plusz üres cellák minden sorban 
 & \cellcolor{blue!30} \\
b_1 & \cellcolor{blue!30} b_2 & \cdots & b_r  \\
 & \cellcolor{blue!30} \\
\end{pmatrix}.
\end{equation}
{\color{gray} Ha nem csalunk láthatatlan dolgokkal, a következőképp nézne ki a két mátrix -- szintén helyes megoldás, csak vizuálisan vektornak látszanak.
\[ A = \begin{pmatrix}
a_1^T \\
\cellcolor{red!30} a_2^T \\ 
\vdots \\
a_m^T 
\end{pmatrix}, 
\quad 
B = \begin{pmatrix}
b_1 & \cellcolor{blue!30} b_2 & \cdots & b_r 
\end{pmatrix}. \]
}
Mind $a_i$, mind $b_j \in \mathbb{R}^{\textcolor{red}{n}}$, így az $a_i^T b_j$ skalár szorzat létezik. 
Az $A \cdot B$ mátrix szorzatot a következőképpen definiáljuk:
\begin{equation}
A \cdot B := \begin{pmatrix}
a_1^T b_1 & \cellcolor{blue!15}a_1^T b_2 & \cdots & a_1^T b_r \\
\rowcolor{red!15} a_2^T b_1 & \cellcolor{purple!30}a_2^T b_2 & \cdots & a_2^T b_r \\
\vdots & \cellcolor{blue!15}\vdots & \ddots & \vdots \\
a_m^T b_1 & \cellcolor{blue!15}a_m^T b_2 & \cdots & a_m^T b_r 
\end{pmatrix}
\in \mathbb{R}^{m \times r}
\end{equation}
\end{defin}

\begin{tet}[Mátrix szorzás tulajdonságai]
A mátrix szorzás asszociatív, de nem kommutatív művelet.
\end{tet}

\begin{tet}[Mátrix szorzat determinánsa]
Ha $A,b\in \mathbb{R}^{n\times n}$ négyzetes mátrixok,
akkor
\begin{equation}
%% \det beépített függvény
\det(A\cdot B) = \det(A) \cdot \det(B).
\end{equation}
\end{tet}

\chapter{Pszeudokód}

\hulipsum[1]

\begin{algorithm} %%csak az úszó környezet
\caption{Lineáris keresés}
\begin{algorithmic}[3] %%környezet kód írására -- vigyázzunk, nem is hasonlít a csomag nevére
%% opcionális argumentum: hány soronként számozzon
\Procedure{linsearch}{A,ertek,@index}
\Require A tömb, ertek a keresett érték
\Ensure index (az első) A-beli index, hogy $A_i$ = ertek, vagy érvénytelen (0)
\State i $\gets$ 1 %%ne felejtsük el a \State parancsokat -- ők gondoskodnak a megfelelő behúzásról a blokkokban!
%% beépített \and parancs nincs kézzel vastagítjuk
\While{$A_i \neq$ ertek \textbf{and} i < Hossz[A]}
\State \Call{inc}{i} %%függvényhívás -- \Function parancs függvényblokkot kezd!
\EndWhile %% ne felejtkezünk el az \End... parancsokról sem!
\If{i < Hossz[A]}
\State index $\gets$ i
\Else
\State index $\gets$ 0
\EndIf
\State \Return index
\EndProcedure
\end{algorithmic}
\end{algorithm}

\hulipsum

\chapter{Programozás}

szöveg előtte... nem teszünk új bekezdést, hogy lássuk, a parancs gondoskodik róla!
\centerpair{bal}{jobb}
szintén új bekezdés nélkül írunk, de automatikusan új bekezdésben (kéne) lennünk

\centerpair{na még egyszer használjuk...}{teszteljük a parancsunkat egy kicsivel hosszabb szövegekkel is\par ez egy új bekezdés. láthatjuk a tördelést és a felülre igazítást is!}

\chapter{Ciklus -- bónusz feladat}

\newcounter{index} %%0-ra incializál, ez jó lesz induláshoz
\newcounter{hatvany}
\setcounter{hatvany}{1} %%2^0 = 1
\whiledo{\value{index}<21}% %%\value: érték lelkérdezése számláló változóként való használatához -- fontos, különben hibát dob az összehasonlítás!
{% %%ciklusmag
\theindex:~\thehatvany\\% %%kiíratás és új sor
%%léptetés:
\stepcounter{index}% %%index++
\setcounter{hatvany}{\numexpr 2*\value{hatvany}\relax}%
%%\numexpr + \relax: ha számolni kell a kifejezést, amit étékül akarunk adni, tegyük ezek közé
}

\end{document}