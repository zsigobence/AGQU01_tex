\documentclass{article}

\usepackage[magyar]{babel}
\usepackage{t1enc}
\usepackage{amsthm}


\newtheorem*{megj}{Megjegyzés}
\theoremstyle{plain}
\newtheorem{tet}{Tétel}
\newtheorem{lemma}[tet]{Lemma}
\theoremstyle{definition}
\newtheorem{defin}{Definíció}
\theoremstyle{remark}
\newtheorem{fel}{Feladat}[section]


\begin{document}
\section{Első}
\begin{tet}[Pitagorasz]
\[a^{2} +b^{2} = c^{2}\]
\end{tet}

\begin{proof}
bizonyítás
\end{proof}

\begin{fel}
Feladat1
\end{fel}

\begin{defin}
A leghosszabb oldalának a négyzete egyenlő a másik két oldal négyzetének összegével.
\end{defin}

\begin{fel}
Feladat
\end{fel}

\begin{tet}
Két nemnegatív szám mértani közepének a két szám szorzatának négyzetgyökét nevezzük.
\end{tet}
\begin{lemma}
ez egy Lemma
\end{lemma}
\section{Második}
\begin{defin}
A mértani közepet szokás geometria középnek is nevezni, és “G” betűvel jelölni.
\end{defin}

\begin{fel}
Feladat3
\end{fel}

\begin{megj}
Megjegyzés
\end{megj}

\end{document}